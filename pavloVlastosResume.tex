%%%%%%%%%%%%%%%%%%%%%%%%%%%%%%%%%%%%%%%%%%%%%%%%%%%%%%%%%%%%%%%%%%%%%%%%%%%%%%%
% Settings
\documentclass[paper=a4,fontsize=11pt]{article} % KOMA-article class

\usepackage[english]{babel}     % Language
\usepackage{amsmath}
\usepackage{fontspec}
\usepackage{graphicx}           % Enable pdflatex
\usepackage{enumitem}           % For eliminating white space for itemize
\usepackage{nopageno}           % Get rid of page number
\usepackage{longtable}\setlength{\LTpre}{0pt}\setlength{\LTpost}{0pt}
\usepackage{pgf}                % For simple math calculations in LaTeX
\usepackage[margin=0.05 \textwidth]{geometry} % Set margins

\setmainfont[Mapping=text-tex]{Calibri}
               
\def \sectionSpace   {0.006\textheight} % Space between sections
\def \subSectionSpace{0.003\textheight} % Space between sections
\def \leftColSpace      {0.1\textwidth} % Space for left column
\def \bulletSpace  {0.05\textwidth}     % Space for bullet point column
\def \restOfColSpace {0.85\textwidth}   % Essentially textwidth - left column space - bullet point space

\def \middleColSpace    {0.5\textwidth}% Space for middle column
\def \bigMiddleColSpace {0.8\textwidth} % Space for middle column

\def \rightColSpace {0.3\textwidth}    % Space for right column
\def \lineThickness {1pt}               % Line thickness for separating

%%%%%%%%%%%%%%%%%%%%%%%%%%%%%%%%%%%%%%%%%%%%%%%%%%%%%%%%%%%%%%%%%%%%%%%%%%%%%%%
\begin{document} 

    %%%%%%%%%%%%%%%%%%%%%%%%%%%%%%%%%%%%%%%%%%%%%%%%%%%%%%%%%%%%%%%%%%%%%%%%%%%
    % Name and Contact
    \noindent
    \begin{minipage}[b]{0.6\textwidth}
    \noindent \\
    \noindent \\
    \noindent {\fontsize{30}{36}\selectfont \textbf{Pavlo Vlastos}}
    \noindent \\
    \end{minipage}
    \begin{minipage}[b]{0.1\textwidth}
    \noindent \textbf{E-mail}:\\
    \noindent \textbf{Website}:\\
    \noindent \textbf{Phone}:\\
    \noindent \textbf{Location}:
    \end{minipage}
    \begin{minipage}[b]{0.4\textwidth}
    \noindent pvlastos@ucsc.edu\\
    \noindent www.pavl.org, github.com/PavloGV\\
    \noindent 1 (307) 797-0475\\
    \noindent Santa Cruz, California
    \end{minipage}
    \vspace{\sectionSpace}
    \noindent\rule{\textwidth}{\lineThickness}

    %%%%%%%%%%%%%%%%%%%%%%%%%%%%%%%%%%%%%%%%%%%%%%%%%%%%%%%%%%%%%%%%%%%%%%%%%%%
    % Quick Summary
    \vspace{-\sectionSpace}
    \begin{longtable}[l]{p{\leftColSpace} p{\restOfColSpace}}
        Summary & 
        % Robotics engineer capable of software developement for embedded systems and control theory. Positive work mentality, strong communication, and teamwork skills. 
        Positive attitude and 
        passionate about solving problems.\\
    \end{longtable}
    \vspace{-\sectionSpace}
    \noindent\rule{\textwidth}{\lineThickness}

    %%%%%%%%%%%%%%%%%%%%%%%%%%%%%%%%%%%%%%%%%%%%%%%%%%%%%%%%%%%%%%%%%%%%%%%%%%%
    % Education

    %%%% Phd Student
    \begin{longtable}[l]{p{\leftColSpace} p{\middleColSpace} p{\rightColSpace}}
        Education & \textbf{Graduate Student in PhD Program} \quad University of California Santa Cruz & \textbf{Oct 2018 - Completion Goal: June 2020} \\
    \end{longtable}
    \begin{longtable}[l]{p{\leftColSpace} p{0.0\textwidth} p{\restOfColSpace}}
        & \textbullet\ &  Computer Engineering with an emphasis in Robotics \& Control\\
        & \textbullet\ &  Researching autonomous surface vehicles using optimal estimation, optimal control, quantized measurements for observation and tracking of propagating surface phenomena.\\
        & \textbullet\ & Designing custom CAN node PCBs for CAN-bus-controlled autonomous boat\\
        & \textbullet\ &  Designing feedback control systems for boat actuators\\
        & \textbullet\ & Implementing oceanographic sensor platform in C, C++\\
        & \textbullet\ & Simulating dynamics in MATLAB\\
    \end{longtable}
    \vspace{\sectionSpace}

    %%%% BS Robotics Engineering
    \begin{longtable}[l]{p{\leftColSpace} p{\middleColSpace} p{\rightColSpace}}
        & \textbf{BS in Robotics Engineering} \quad The University of California at Santa Cruz & \textbf{Oct 2013 - June 2017} \\
    \end{longtable}
    \begin{longtable}[l]{p{\leftColSpace} p{0.0\textwidth} p{\restOfColSpace}}
        & \textbullet\ &  Emphasis in digital feedback control\\
    \end{longtable}
    \vspace{-\sectionSpace}
    \noindent\rule{\textwidth}{\lineThickness}

    %%%%%%%%%%%%%%%%%%%%%%%%%%%%%%%%%%%%%%%%%%%%%%%%%%%%%%%%%%%%%%%%%%%%%%%%%%%
    % Work Experience

    %%%% Inboard Technologies
    \begin{longtable}[l]{p{\leftColSpace} p{\middleColSpace} p{\rightColSpace}}
        Work &  \textbf{Intern} \quad Inboard Technologies  & \textbf{Summer 2017} \\
    \end{longtable}
    \begin{longtable}[l]{p{\leftColSpace} p{0.0\textwidth} p{\restOfColSpace}}
        Experience & 
        \textbullet\  & Tested and repaired lithium-ion batteries used for powering brush-less DC motors.\\ 
        & \textbullet\  & Designed, built, and programed a Dynamo-meter test-rig for board testing. Test-rig measured motor torque, board vibration, and simulated acceleration.\\
    \end{longtable}

    %%%% UC Santa Cruz
    \begin{longtable}[l]{p{\leftColSpace} p{\middleColSpace} p{\rightColSpace}}
        % Work
        & \textbf{Teaching Assistant} \quad UC Santa Cruz & \textbf{Oct 2017 - June 2019} \\
    \end{longtable}
    \begin{longtable}[l]{p{\leftColSpace} p{0.0\textwidth} p{\restOfColSpace}}
        % Experience
        &\textbullet\ & \textbf{Mechatronics}: Managed staff of 10 tutors in leading a lab of 60-90 students through the most difficult course in the UCSC school of engineering. Designed as “Engineering Boot-Camp”, student teams design and build an autonomous robot which perform a series of perception and actuation tasks. Students program a hierarchical state machine (HSM) in C and integrate a series of sensors and actuators culminating in a course competition.\\
        & \textbullet\ & \textbf{Sensing and Sensor Technology}: Students are taught how to interface with digital sensors and other peripherals. Subject content includes calibration and implementation of inertial measurment units, encoders, and ping sensors. Students write C code on PIC32 microcontroller to digitally filter out noise and calibrate sensors.\\
        & \textbullet\ & \textbf{Embedded C Programming on Microcontrollers}: Students are taught C syntax, data structures, interrupt subroutines, sorting algorithms, sensor debouncing, and system integration.\\
    \end{longtable}
    \vspace{-\sectionSpace}
    \noindent\rule{\textwidth}{\lineThickness}

    %%%%%%%%%%%%%%%%%%%%%%%%%%%%%%%%%%%%%%%%%%%%%%%%%%%%%%%%%%%%%%%%%%%%%%%%%%%
    % Projects

    %%%% Autonomous Boat
    \begin{longtable}[l]{p{\leftColSpace} p{\middleColSpace} p{\rightColSpace}}
        Projects & \textbf{Autonomous Boat} \quad PhD Project & \textbf{Spring 2018 - current} \\
    \end{longtable}
    \begin{longtable}[l]{p{\leftColSpace} p{0.0\textwidth} p{\restOfColSpace}}
        & \textbullet\ &  Utilize CAN bus, custom CAN-node pcb designs and PIC32\\
        & \textbullet\ & $\text{L}_2^+$ guidance algorithm\\
        & \textbullet\ & Sensor fusion of GPS and IMU\\
        & \textbullet\ & Uncertainty suppression\\
    \end{longtable}
    \vspace{\sectionSpace}
    %%%% Ball Balancing Robot
    \begin{longtable}[l]{p{\leftColSpace} p{\middleColSpace} p{\rightColSpace}}
         & \textbf{Ball Balancing Robot } \quad Senior Design Project & \textbf{Jan 2017 - May 2017} \\
    \end{longtable}
    \begin{longtable}[l]{p{\leftColSpace} p{0.0\textwidth} p{\restOfColSpace}}
        & \textbullet\ & Developed a robot capable of balancing on top of a ball.\\
        & \textbullet\ & Programmed PID controllers in C for closed-loop motor torque control.\\
        & \textbullet\ & Implemented multiple system models in testing.\\
        & \textbullet\ & Robot used IMU and motor encoders for feedback.\\
    \end{longtable}
    \vspace{\sectionSpace}
    %%%% Mechatronics Robot
    \begin{longtable}[l]{p{\leftColSpace} p{\middleColSpace} p{\rightColSpace}}
         & \textbf{Fully Autonomous Robot} \quad Mechatronics Class & \textbf{Jan 2017 - May 2017} \\
    \end{longtable}
    \begin{longtable}[l]{p{\leftColSpace} p{0.0\textwidth} p{\restOfColSpace}}
        & \textbullet\ & Created tape-following robot programmed in C, using hierarchical state machine (HSM).\\
        & \textbullet\ & Designed and soldered sensors. Used event driven programming.\\
        & \textbullet\ & Robot detected tape, 2.0kHz infrared, and 27kHz track-wire.\\
        & \textbullet\ & Robot navigated to ping-pong balls, loaded balls, and shot target.\\
    \end{longtable}
    \vspace{-\sectionSpace}
    \noindent\rule{\textwidth}{\lineThickness}

    %%%%%%%%%%%%%%%%%%%%%%%%%%%%%%%%%%%%%%%%%%%%%%%%%%%%%%%%%%%%%%%%%%%%%%%%%%%
    % Skills
    \begin{longtable}[l]{p{\leftColSpace} p{0.0\textwidth} p{\restOfColSpace}}
    Skills  & \textbullet\  &\textbf{Software Engineering}: Embedded C, C++, Python, LC3 Assembly, MIPS, MATLAB, Event-Driven Programming, Hierarchical state machines, Linux.\\ 
            & \textbullet\  & \textbf{Robotics}: Kalman Filters, Optimal Estimation, Attitude Estimation, Feedback Control, Optimal Control, (Inverse) Kinematics, Denavit-Hartenberg parameters, Convex Optimization, Dynamic Programming, Linear Programming, Sensor fusion.\\ 
            & \textbullet\  & \textbf{Embedded Systems}: CAN, I2C, UART, SPI, ADC, DAC, sensor integration.\\ 
            & \textbullet\  & \textbf{Computer Engineering}: Computer Architecture, Logic Design.\\ 
            & \textbullet\  & \textbf{Algorithms}: A-star search, Alpha-Beta Pruning, Merge Sort, Binary Search.\\
            & \textbullet\  & \textbf{Design Tools}: SolidWorks, Eagle CAD, Fusion 360, MPLABX IDE.\\ 
            & \textbullet\ & \textbf{Hardware Sensors \& Actuators}: Inertial Measurement Units (IMUs), Gyros, Encoders, Flex Sensors, Ping Sensors, Thermistors (NTC), Capacitive touch sensors, Infrared Radiation sensors, Track-wire sensors, brushed \& brush-less DC motors, AC motors, servos, Buck-Boost regulators, dynamo-meter.\\
            % & \textbullet\ & \textbf{Additional Keywords}: Gauss-Markov Processes, Lyapunov Stability, Quantized measurements, Least-Squares.\\
    \end{longtable}
\end{document}
%%%%%%%%%%%%%%%%%%%%%%%%%%%%%%%%%%%%%%%%%%%%%%%%%%%%%%%%%%%%%%%%%%%%%%%%%%%%%%%